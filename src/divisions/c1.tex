\chapter{热力学}

\section{导言}

	\quoteParagraph{\termIntroduce{热力学}是一种对热平衡下的宏观系统的性质的唯象描述。}

	设想你是一个后\humanNameTranslation{牛顿}{Newton}时代的物理学家,正试图理解如一个容器中的气体这样的简单系统的行为。你会如何推进工作?
	一个成果的物理理论的原型是经典力学,它从简单的基本定律出发并应用微积分这一数学装置来描述了错综复杂的运动。
	类比地,你可能如下以如下的方式推进:

	\begin{itemize}
		\item 尽可能地理想化所要研究的\termIntroduce{系统}(正如点粒子模型那样)。这系统上的机械功的概念我们当然是熟悉的,但它可能会由于热交换而变得很复杂。对其的对策首先是要检查所谓\termIntroduce{封闭系统},它由不容许与环境的任何热交换的所谓\termIntroduce{绝热壁}所隔离。当然,我们最终也需要研究能通过\termIntroduce{透热壁}来与外部世界交换热量的\termIntroduce{开放系统}。
		\item 由于点粒子的状态是由其坐标(和动量)量化的,宏观系统的性质也可以由一定数量的\termIntroduce{热力学坐标}或\termIntroduce{状态函数}来描述。其中我们最熟悉的坐标是与机械功相关的那些,如(流体的)压强与体积、(薄膜的)表面张力与面积、(线的)张力与长度、(电介质的)电场与极化等。我们还将会看到,还有额外的一些与机械功无关的状态函数。这些状态函数仅当系统处于所谓\termIntroduce{平衡态}时才是良定义的,平衡状态的性质在我们所关心的时间尺度(观测时间)上没有\shortRadicalTranslation{可见}{appreciably}的变化。由于对\bracketedText{观测时间}这一概念的依赖,\bracketedText{平衡态}的概念是有些主观的。例如,窗玻璃在几十年内作为固体处于平衡状态,但在几千年的时间尺度上却像流体一样流动。 在另一个极端,考虑大爆炸最初几分钟内的早期宇宙中的物质与辐射之间的平衡是完全合理的。
		\item 最后,状态函数之间的关系由热力学定律来描述。作为一种唯象描述,这些定律基于大量的经验观察。然后一个\shortRadicalTranslation{融贯}{coherent}的逻辑和数学结构便在这些观察的基础上构筑而成,进而带来各种有用的概念以及各种物理量间的可检验的关系。热力学定律只能由更基本的(微观的)自然理论来证明。例如,统计力学试图从粒子集合的演化的经典的或量子的\shortRadicalTranslation{动力学}{mechanical}方程出发来取得这些定律。
	\end{itemize}

\section{第零定律}

	热力学第零定律描述了热平衡的传递性这一本质特性。它说:

	\quoteParagraph{%
		若两个系统A与B都分别与第三个系统C\longRadicalTranslation{处于共同平衡}{in equilibrium with},则它们两个也处于共同平衡。%
	}

	尽管第零定律表面上很简单,但它蕴含了一个重要的状态函数——\termIntroduce{经验温度}\(\Theta\)——的存在性,处于共同平衡的各系统具有相同的温度。

	% Graphics

	令系统\(A,B,C\)的平衡态由\(\{A_1,A_2,\cdots\},\{B_1,B_2,\cdots\},\{C_1,C_2,\cdots\}\)分别描述。$A$与$C$处于共同平衡的假定将蕴含了一个$A,C$的坐标之间的约束,即,\(A_1\)的改变必然伴随着\(\{A_2,\cdots;C_1,C_2,\cdots\}\)的某些改变,才能得以维持\(A,C\)的共同平衡。

	将此约束记作
	\begin{equation}
		f_{AC}(A_1,A_2,\cdots;C_1,C_2,\cdots)=0.
	\end{equation}
	$B,C$的平衡蕴含着一个类似的约束
	\begin{equation}
		f_{BC}(B_1,B_2,\cdots;C_1,C_2,\cdots)=0.
	\end{equation}
	注意到我们假定每个系统各自处于自身的力学平衡中。如果它们也被允许有某种共同平衡,那么我们就需要一些附加条件(如,常压)来描述它们的\shortRadicalTranslation{联合力学平衡}{joint mechanical equilibrium}。

	我们显然能以多种不同方式表述上述约束书。例如,我们可以研究\(C_1\)的由其他参量的变化引起的变化。这等价于对上述每个方程求解\(C_1\)而得到\footnote{从纯数学的角度来看,不一定可以从任意的约束条件中解出\(C_1\)。然而,由于约束描述真实物理参量的要求清楚地表明我们可以取得作为其余参量的函数的\(C_1\)。}
	\begin{equation}
		\begin{aligned}
			C_1=F_{AC}(A_1,A_2,\cdots;C_2,\cdots),\\
			C_1=F_{BC}(B_1,B_2,\cdots;C_2,\cdots).
		\end{aligned}
	\end{equation}
	因此如果\(C\)分别与$A,B$处于共同平衡,我们就必然有
	\begin{equation}\label{eq:1.4}
		F_{AC}(A_1,A_2,\cdots;C_2,\cdots)=F_{BC}(B_1,B_2,\cdots;C_2,\cdots).
	\end{equation}
	然而,根据第零定律,在\(A,B\)之间还有一个平衡,它蕴含了约束
	\begin{equation}\label{eq:1.5}
		f_{AB}(A_1,A_2,\cdots;B_1,B_2,\cdots)=0.
	\end{equation}
	我们可以选择任意一组满足上式的参量\(\{A, B\}\)代入\ref{eq:1.4}式。由此产生的等式必须完全独立于该等式中的任何一组变量\(\{C\}\)而保持成立。然后我们可以改变这些参数,沿着\ref{eq:1.5}式所约束的流形移动,而这时无论$C$的状态如何,\ref{eq:1.4}式都将保持有效。因此,必然可以消去$C$的坐标来简化\ref{eq:1.4}式。或者说,我们可以任意选取一个固定的参数集$C$,并从此忽略它们,将\(A,B\)的平衡条件\ref{eq:1.5}式简化为
	\begin{equation}
		\Theta_A(A_1,A_2,\cdots)=\Theta_B(B_1,B_2,\cdots),
	\end{equation}
	也就是说,平衡由一个记作$\Theta$的热力学坐标的函数来标识。
	这个函数指定了状态方程,而$A$的等温线由\(\Theta_A(A_1,A_2,\cdots)=\Theta\)这一条件给出。此时\(\Theta\)有许多可能的选法,不过我们这里所要强调的是约束着热平衡中的各系统的参量的函数的存在性。

	\(\Theta\)这个量与力学系统中的力有一种相似性。
	考虑两个可以互相做功的一维系统,比如两个连接起来的弹簧。
	当每个物体施加在另一个物体上的力相等时,就会达到平衡。
	(当然,不同于温度这个标量,作为矢量的力有一个方向,我们忽略了这里的一些复杂事项。气体活塞的压强会是一个更适当的类比对象。)
	多个这样的物体之间的力学平衡同样具有传递性,我们可以从后者出发来推断一个机械力的存在。

	作为一个例子,我们考虑如下三个系统
	\begin{enumerate}[label=(\Alph*)]
		\item 长度为$L$、张力为$F$的绳索
		\item 磁场$B$中的磁化强度为$M$的顺磁体
		\item 压强$P$下的体积为$V$的气体
	\end{enumerate}

	% Graphics

	观察表明若这些系统处于平衡中,它们的坐标就会满足下列约束:
	\begin{equation}
		\begin{aligned}
			\left(P+\frac{a}{V^2}\right)(V-b)(L-L_0)-c[F-K(L-L_0)]=0,&\\
			\left(P+\frac{a}{V^2}\right)(V-b)M-dB=0.&
		\end{aligned}
	\end{equation}
	这两个条件可以组织成三个经验温度函数:
	\begin{equation}
		\Theta\propto\left(P+\frac{a}{V^2}\right)(V-b)=c\left(\frac{F}{L-L_0}-K\right)=d\frac{B}{M}.
	\end{equation}
	应注意到是第零定律严格地限制了描述两体平衡的约束方程的形式。若是一个任意的函数,是未必能组织成两个经验温度函数间的等式的。

	我在上例中选取的约束实际上复现了本书后面将遇到并讨论的三个著名的状态方程。他们有一个我们更熟悉的形式,写作:
	\begin{equation}
		\left\{\begin{aligned}
			(P+a/V^2)(V-b)=Nk_BT&\qquad\text{({\vanDerWaals}气体)}\\
			M=(N\mu_B^2B)/(3k_BT)&\qquad\text{(\humanNameTranslation{居里}{Curie}顺磁体)}\\
			F=(K+DT)(L-L_0)&\qquad\text{(弹性体的\humanNameTranslation{胡克}{Hooke}定律)}
		\end{aligned}\right..
	\end{equation}
	注意到我们用到了温度$T$的符号,而非更一般的经验温度$\Theta$。这一具体温标可以根据理想气体的性质构造出来。

	\termIntroduce{理想气体温标}:虽然第零定律仅表明了等温线的存在,但要在现阶段建立实用的温标,一个参照的系统是必要的。\termIntroduce{理想气体}在热力学中占有重要地位并提供这个必要的参照。经验观察表明,对于任一种充分稀薄的气体,压强和体积的乘积沿等温线不变。
	所谓理想气体指的就是真实气体的这一\textit{稀薄}极限;而理想气体温度正比于这一乘积。
	该常值比例系数由水的固液气三相点的温度为参照来确定,该温度在1954年的第10届的国际计量大会上被设为\(273.16K\)。
	使用一稀薄(也就是说如$P\to0$那样)的气体作为温度计,一个系统的温度便可得自
	\begin{equation}
		T(K)\equiv 273.16\times\left(\lim_{P\to0}(PV)_{\text{系统}}/\lim_{P\to0}(PV)_{\text{冰-水-蒸汽}}\right).
	\end{equation}

\section{第一定律}

	处理简单的力学系统时,能量守恒是一个重要原则。例如,一个粒子的位置可以因外部功的势而改变,而这变化又关联于势能的变化。实验观测表明,若系统是\termIntroduce{适当隔离}了的——也就是说如果唯一的能量来源是机械的,就有一条类似的原理在宏观物体层面起效。我们将对这些观测结果使用如下的表述:

	\quoteParagraph{%
		改变绝热孤立系统状态所需的功的数量仅取决于初态和末态,而不依赖于做功的方式或系统所经历的中间阶段。%
	}

	% Graphics

	对于在势中运动的粒子,所需的功可用于描绘势能的\shortRadicalTranslation{地形}{landscape}\translationNote{指状态集到实数的一个映射\(f:\mathbf X\to\mathbb R\)。}。类似地,对于热力学系统我们可以构造另一个状态函数,即内能\(E(\mathbf X)\)。\(E(\mathbf X)\)可以从由初态\(\mathbf X_i\)到末态\(\mathbf X_f\)的\termIntroduce{绝热}转换所需的功 W 中根据
	\begin{equation}
		\Delta W=E(\mathbf X_f)-E(\mathbf X_i)
	\end{equation}
	来确定(至多相差一个常数)。

	另一组观察表明,一旦去除绝热约束,功的量值就不再等于内能的变化量。差值\(\Delta Q=\Delta E−\Delta W\)定义为系统从其环境中吸收的热量。显然,在这样的转换中,\(\Delta Q, \Delta W\)无法单独构成状态函数,因为它们取决于如做功方式这样的外部因素,而不仅依赖于末态。
	为了强调这一点,对于一个\shortRadicalTranslation{无穷小}{differential}的转换,我们这样写道:
	\begin{equation}
		\ddbar Q=\dd E-\ddbar W,
	\end{equation}
	其中\(\dd E=\sum_i\pd_iE\dd X_i\)可以通过微分得到,\(\ddbar Q,\ddbar W\)则一般不能。
	另外应注意我们的正负符号约定,功和热的符号表示增加到系统中的能量,而不是反过来。
	因此,热力学第一定律指出,要改变系统的状态,能量须取定值——而它可以是机械功或热量的形式。
	这也可以被视为一种定义和量化热量交换的方法。

	\termIntroduce{准静态}转换是一种进行得足够慢从而系统总是处于平衡状态的转换。因此,在过程的任何阶段,系统的热力学坐标都存在且原则上可计算。对于此类变换,对系统所做的功(与系统所做的功大小相等但符号相反)可能与这些坐标的变化有关。
	作为一个力学上的例子,现在考虑一个弹簧或橡皮筋的拉伸。
	为了将系统的势能构建为其长度$L$的函数,我们可以足够慢地拉动弹簧,以便在每个阶段外力都与弹簧的内力$F$相匹配。
	对于这样的准静态过程,弹簧势能的变化为\(\int F\dd L\)。如果对弹簧的牵拉较猛,一些外部功将转化为动能,并最终随弹簧的逐渐休止而耗尽。

	我们可以从上面的例子出发来推广,将状态函数\(\{\mathbf X\}\)分割为一组\termIntroduce{广义位移}\(\{\mathbf x\}\)以及与其共轭的\termIntroduce{广义力}\(\{\mathbf J\}\),使得它们对于无穷小的准静态变换满足\footnote{我用符号\(\{\mathbf J\}\)而非\(\{\mathbf F\}\)来指示力,从而将后者留给自由能。我希望读者不会将其同(在本书中很少出现的)电流(有时也用\(\{\mathbf J\}\)来指示)混淆。}
	\begin{equation}
		\ddbar W=\sum_iJ_i\dd x_i.
	\end{equation}

	\begin{table}[H]\label{tab:1.1}
		\centering\begin{tabularx}{\textwidth}{XX>{\raggedleft\arraybackslash}XX>{\raggedleft\arraybackslash}X}
			\toprule\midrule
			系统 & 力 & & 位移 & \\\midrule
			绳索 & 张力 & $F$ & 长度 & $L$\\
			薄膜 & 表面张力 & $F$ & 面积 & $A$\\
			流体 & 压强 & $-P$ & 体积 & $V$\\
			磁体 & 张力 & $H$ & 磁化强度 & $M$\\
			电介质 & 张力 & $E$ & 极化强度 & $P$\\
			化学反应 & 化学势 & $\mu$ & 粒子数 & $N$\\
			\midrule\bottomrule
		\end{tabularx}
		\caption{广义力与广义位移}
	\end{table}

	表\ref{tab:1.1}给出了这样的共轭坐标的一些常见例子。
	注意我们约定压强$P$是由系统对壁施加的力(而非系统所受到的、方向与此相反得到力)来计算的。
	流体静力学功的负号就是从这里来的。

	这些位移通常是\termIntroduce{广延量},即与系统\shortRadicalTranslation{大小}{size}成正比,而力是\termIntroduce{强度量},与大小无关。后者是衡量平衡态的指标;例如,压强在处于平衡态的气体中(在没有外部势的情况下)是均匀的,并且互相接触且平衡的两团气体的压强是相等的。如在第零定律的有关讨论所说的那样,温度在热交换问题中起着类似的作用。
	那么是否有与温度对应的位移?如果有的话是什么?
	这个问题将在随后的章节中得到解答。

	\termIntroduce{理想气体}:我们在讨论第零定律时就能注意到,理想气体的状态方程具有特别简单的形式,即\(PV\propto T\)。理想气体的内能也采用非常简单的形式,从\termIntroduce{\humanNameTranslation{焦耳}{Joule}自由膨胀实验}能观察到这一点:
	测量表明,如果理想气体绝热地(但未必准静态地)膨胀,体积从\(V_i\)变为\(V_f\),而初始温度和最终温度相同。由于该转换是绝热的\(\Delta Q=0\),并且系统没有外部做功 \(\Delta W=0\),因此气体的内能没有变化。
	由于气体的压强和体积在此过程中发生变化而温度却没有变化,因此我们得出内能仅取决于温度的结论,即\(E(V,T)=E(T)\)。理想气体的这一性质实际上是其状态方程形式的结果,这一点将应在本章末的一个习题中得到证明。

	% Graphics

	\termIntroduce{响应函数}是刻画系统宏观行为特征的惯常方法。它们是借助外部\shortRadicalTranslation{探头}{probe}而从热力学坐标的变化来在实验上测得的。一些常见的响应函数如下。

	\termIntroduce{热容}是得自向系统输入热量时的温度变化量。由于热量不是状态函数,因此还必须指定输热的路径。例如,对于气体我们可以计算在恒定体积或恒定压强下的热容,分别表示为\(C_V=\ddbar Q/\dd T|_V, C_P=\ddbar Q/\dd T|_P\)。后者会更大,因为一些热量在体积变化时的做功中消耗掉了:

	\begin{equation}\label{eq:1.14}
		\begin{aligned}
			&C_V=\left.\frac{\ddbar Q}{\dd T}\right|_V=\left.\frac{\dd E-\ddbar W}{\dd T}\right|_V=\left.\frac{\dd E+P\dd V}{\dd T}\right|_V=\left.\frac{\pd E}{\pd T}\right|_V,\\
			&C_P=\left.\frac{\ddbar Q}{\dd T}\right|_P=\left.\frac{\dd E-\ddbar W}{\dd T}\right|_P=\left.\frac{\dd E+P\dd V}{\dd T}\right|_P=\left.\frac{\pd E}{\pd T}\right|_P+P\left.\frac{\pd V}{\pd T}\right|_P.
		\end{aligned}
	\end{equation}
	
	\termIntroduce{力常数}所测量的是位移对力的(无穷小)比值,是对弹簧常量的推广。力常数的例子如等温压缩率\(\kappa_T=-\pd V/\pd P|_T/V\)、磁化率\(\chi_T=\pd M/\pd B|_T/V\)。从理想气体状态方程\(PV\propto T\)中,我们可以得到\(\kappa_T=1/P\)。

	\termIntroduce{热响应}探测的是热力学坐标由温度引起的变化量。例如,气体膨胀率由\(\alpha_P=\pd V/\pd T|_P/V\)给出,对于理想气体而言它等于\(1/T\)。

	由于理想气体的内能只取决于它的温度,\(\thermoRatioI{E}{T}{V}=\thermoRatioI{E}{T}{P}=\dd E/\dd T\),从而\ref{eq:1.14}式简化为
	\begin{equation}
		C_P-C_V=P\thermoRatio{V}{T}{P}=PV\alpha_P=\frac{PV}{T}\equiv Nk_B.
	\end{equation}
	最后一个等号来自广延性:对于给定量的理想气体,常量\(PV/T\)正比于气体中的粒子数$N$;该比值即是{\boltzmann}常量,其值为\(k_B\approx1.4\times10^{−23}JK^{−1}\)。

\section{第二定律}

	19世纪的热力学科学发展\shortRadicalTranslation{在实践上的}{practical}推动力是热机的出现。在工业革命期间,对机器做功的依赖的增加要求我们更好地理解热能到功的转化的基本原理。了解对引擎效率等的实用的考量如何为我们带来了熵等抽象概念是非常有趣的。

	一个理想化的\termIntroduce{热机}的工作过程是:从热源(例如煤火)吸收一定量的热量\(Q_H\),然后将其中一部分转化为功$W$,并将剩余的热量\(Q_C\)排放到散热器(例如大气)中。热机的效率由下式计算
	\begin{equation}
		\eta=\frac{W}{Q_H}=\frac{Q_H-Q_C}{Q_H}\leq1.
	\end{equation}
	一个理想化的\termIntroduce{\shortRadicalTranslation{制冷机}{refrigerator}}像是一台反向运作的引擎:它借助功$W$来从低温系统提取热量$Q_C$,而在更高的温度下排放热量$Q_H$。我们可以类似地为制冷剂的性能而如下地定义一个品质因数的量
	\begin{equation}
		\omega=\frac{Q_C}{W}=\frac{Q_C}{Q_H-Q_C}
	\end{equation}

	% Graphics

	第一定律排除了所谓的\termIntroduce{第一类永动机},即无需消耗任何能量即可产生功的机器。然而,通过将水转化为冰来产生功的热机并不违背能量守恒。这样的\termIntroduce{第二类永动机}固然能够解决世界能源问题,但它被热力学第二定律排除了。热力学第二定律的本质就是\bracketedText{热流的自然方向是从较热的物体流向较冷的物体结果}这一观察。第二定律有许多不同的表述,例如下面两个:

	\termDefinition{{\kelvin}表述}{任何一个将热量完全转化为功而无其他影响的过程都是不可能的。}

	\termDefinition{{\clausius}表述}{任何一个使低温物体传热到高温物体而无其他影响的过程都是不可能的。}

	第一个表述排除了完美的发动机,而第二个表述排除了完美的制冷剂。既然我们将使用这两个表述,我们首先证明它们是等价的。对{\kelvin}与{\clausius}表述的等价性的证明是通过指出若其中一个被违反,另一个也必被违反。

	\begin{enumerate}[label=(\alph*)]
		\item 让我们假设有一台机器能将热量$Q$从较冷的区域带到较热的区域而违背{\clausius}表述。现在考虑一台运作于这两个区域间的热机,它从较热的区域吸收热量$Q_H$而于较冷的散热区域排放$Q_C$。它们组合而来的系统从热源获取了\(Q_H-Q\),产生的功为\(Q_H-Q_C\),在散热区域排放\(Q_C-Q\)。如果我们调整热机的输出量而使得\(Q_C=Q\),那么\shortRadicalTranslation{最终}{net}得到的就是一台\(100\%\)效率的热机,这违反了{\kelvin}表述。
		% Graphics
		\item 另一方面,现在假定一个机器能吸取热量$Q$而完全转化为功来违背{\kelvin}表述。其输出的功可以用于驱动一台制冷机,而最终得以从低温物体像高温物体传热,从而违背{\clausius}表述。
		% Graphics
	\end{enumerate}

\section{{\carnot}热机}
	\quoteParagraph{{\carnot}热机是指可逆的、在一个循环中运行、且其所有热交换都在源温度$T_H$和散热温度$T_C$下进行的热机。}

	\termIntroduce{可逆}过程指的是可以通过简单地颠倒其输入和输出来\shortRadicalTranslation{在时间上}{in time}逆转的过程。它是力学中的无摩擦运动在热力学上的等价物。由于时间可逆性意味着平衡,因此可逆转换必然是准静态的,但反之则未必(例如,可能由于摩擦而存在能量耗散)。\termIntroduce{循环}运行的热机在过程结束时总会回到其原始内部状态。{\carnot}热机的标志性特征在于它与周围环境的热交换仅在两个温度下进行。

	第零定律允许我们为这些热交换选择在温度$T_H$和$T_C$的两条等温线。要完成{\carnot}循环,我们须由坐标空间中的可逆绝热路径来把这两条等温线连接起来。由于热量不是状态函数,我们一般不知道如何构造这样的路径。好在现在我们已有足够的信息来构建一个使用理想气体作为其内部工作物质的{\carnot}热机。出于演示的目的,我们现在计算具有内能
	\begin{equation*}
		E=\frac{3}{2}Nk_BT=\frac{3}{2}PV,
	\end{equation*}
	的单原子理想气体的绝热线。

	沿着一条准静态路径,我们有
	\begin{equation}
		\ddbar Q=\dd E-\ddbar W=\dd\left(\frac{3}{2}PV\right)+P\dd V=\frac{5}{2}P\dd V+\frac{3}{2}V\dd P.
	\end{equation}

	由绝热条件\(\ddbar Q=0\),可给出一条路径
	\begin{equation}
		\frac{\dd P}{P}+\frac{5}{3}\frac{\dd V}{V}=0,
		\quad\implies\quad
		PV^\gamma=\text{常量},
	\end{equation}
	其中\(\gamma=5/3\)。

	% Graphics

	绝热线明显不同于等温线,我们可以选择两条这样的曲线来相交于我们的等温线,从而完成卡诺循环。\(E\propto T\)的假定不是必需的,在本章末尾提供的其中一个问题中,你将为一般性的\(E(T)\)构造绝热线。实际上,对于任一\(E(J,x)\)的双参数系统,类似的构造也都是可能的。

	\termDefinition{{\carnot}定理}{%
		任何工作于两个(温度为$T_H,T_C$的)热库间的热机都不能具有比其间的{\carnot}热机更高的效率。
	}

	由于{\carnot}热机是可逆的,它可以反向运作而成为一个制冷机。现在用非{\carnot}热机来逆向驱动{\carnot}热机。
	我们分别用$Q_H,Q_C$和$Q'_H,Q'_C$来表示非{\carnot}热机和{\carnot}热机的热交换。两个热机的总效果是将值为\(Q_H-Q'_H=Q_C-Q'_C\)的热量从$T_H$传递到$T_C$。根据{\clausius}表述,传热的量不能为负,即\(Q_H\geq Q'_H\)。由于此过程中所涉及的功$W$是等值的,我们得知
	\begin{equation}
		\frac{W}{Q_H}\leq\frac{W}{Q'_H},
		\quad\implies\quad
		\eta_{\text{\carnot}}\geq\eta_{\text{非\carnot}}.
	\end{equation}

	% Graphics

	\corollaryParagraph{}{%
		由于可逆\carnot 热机能够逆转地驱动自身,所以它们有着同一个\termIntroduce{普遍}效率\(\eta(T_H,T_C)\)。%
	}

	热力学温标:如前所述,使用理想气体(或任何其他双参数系统)作为工作物质来构造\carnot 热机至少在理论上是可能的。我们现在已知晓,无论用材、设计和构造如何,所有此类循环和可逆热机都具有相同的最大理论效率。由于此最大效率仅取决于两个温度,因此它可用于构造一个温标。这种温标具有独立于材料(例如,理想气体)性质的迷人性质。
	要构造这样一个尺度,我们首先需对\(\eta(T_H,T_C)\)的形式得到一些限制。考虑两个串联运行的\carnot 热机,其一运行于温度$T_1$和$T_2$间,另一个则在$T_2,T_3$(\(T_1>T_2>T_3\))间。分别用\(Q_1,Q_2,W_{12}\)和\(Q_2,Q_3,W_{23}\)来表示两个热机的热交换和功输出。注意,第一个热机排放的热量由第二个热机吸收,因此总效果即是另一个\carnot 热机(因为每个组件都是可逆的),其热交换为\(Q_1,Q_3\)、功输出为\(W_{13}=W_{12}+W_{23}\)。

	% Graphics

	使用普遍效率,可给出下式来将三个热交换量联系起来
	\begin{equation*}
		\left\{
			\begin{aligned}
				Q_2=Q_1-W_{12}=Q_1[1-\eta(T_1,T_2)],\\
				Q_3=Q_2-W_{23}=Q_2[1-\eta(T_2,T_3)]=Q_1[1-\eta(T_1,T_2)][1-\eta(T_2,T_3)],\\
				Q_3=Q_1-W_{13}=Q_1[1-\eta(T_1,T_3)].
			\end{aligned}
		\right.
	\end{equation*}
	比较后两条表达式可知
	\begin{equation}
		Q_1[1-\eta(T_1,T_3)]=Q_1[1-\eta(T_1,T_2)][1-\eta(T_2,T_3)].
	\end{equation}
	这一性质意味着\(1-\eta(T_1,T_2)\)可写为一个比值的形式\(f(T_2)/f(T_1)\),我们约定其为\(T_2/T_1\),也就是说,
	\begin{equation}\label{eq:1.22}
		1-\eta(T_1,T_2)=\frac{Q_2}{Q_1}\equiv\frac{T_2}{T_1},
		\quad\implies\quad
		\eta(T_H,T_C)=\frac{T_H-T_C}{T_H}.
	\end{equation}

	\ref{eq:1.22}式将温度定义到了至多相差一个比例常量的程度,我们再次通过选择水、冰和蒸汽的三相点为$273.16K$来确定这个常量。目前为止,我们已经交替地使用了符号$\Theta$和$T$。事实上,借由完全气体的\carnot 循环,可以证明(见习题)理想气体温标与热力学温标等价。热力学温标在实用的立场看来明显没什么用;它的优势是概念性的,因为它不依赖于具体物质的性质。所有热力学温度都是正的,因为据\ref{eq:1.22}式可知从温度$T$提取的热量正比于$T$。若在负温度存在,则在负温度和正温度之间运行的热机将同时从两个热库中提取热量并将其总和完全转化为功,而这违反第二定律的\kelvin 表述。

\section{熵}
	为了最终能构造出与温度共轭的状态函数,我们求助于以下定理:

	\theoremParagraph{\clausius 定理}{%
		对任何循环\shortRadicalTranslation{过程}{transformation}(无论可逆与否),都有\(\oint\ddbar Q/T\leq 0\),其中\(\ddbar Q\)是在温度\(T\)下向系统供给的热量。%
	}

	将循环细分为一系列无穷小的过程,在这些过程中,系统以热\(\ddbar Q\)和功\(\ddbar W\)的形式接收能量输入。在各个时间区间,系统不必处于平衡状态。将系统的所有热交换导向\carnot 热机的一个端口,而热机在另一端应有着一个温度固定为$T_0$的热库。(可以有多个\carnot 热机,只要它们都有一端连接到$T_0$即可。)\longRadicalTranslation{由于{\xspace\(\ddbar Q\)}是不定号的,\carnot 热机应在两个方向都运行有一系列无穷小循环。}{Since the sign of \(\ddbar Q\) is not specified, the Carnot engine must operate a series of infinitesimal cycles in either direction.}为了在某个阶段向系统输送热量\(\ddbar Q\),热机须从固定热库中提取热量\(\ddbar Q_R\)。若热量被传递到系统的局部温度为$T$的一个部分,则根据\ref{eq:1.22}式,有
	\begin{equation}
		\ddbar Q_R=T_0\frac{\ddbar Q}{T}.
	\end{equation}

	循环完成后,系统和\carnot 热机恢复到原来的状态。整个过程的净效应是从热库提取了热量\(Q_R=\oint\ddbar Q_R\)并将其转化为外部功$W$。功\(W=Q_R\)是\carnot 热机所做的功与系统在整个循环中所做的功之和。根据第二定律的\kelvin 表述,\(Q_R=W\leq 0\),也就是说
	\begin{equation}\label{eq:1.24}
		T_0\oint\frac{\ddbar Q}{T}\leq 0,
		\quad\implies\quad
		\oint\frac{\dd Q}{T}\leq 0,
	\end{equation}
	其中箭头成立是因为\(T_0>0\)。注意,\ref{eq:1.24}式中的$T$所指的是对能在整个循环中定义出温度的那些准静态过程而言的系统整体的温度。否则,它就仅是\carnot 热机的存放部分热量的地方(比如说,在系统的边界处)的局部温度。

	\clausius 定理的一些\emph{后果}:

	\begin{enumerate}
		\item 对于可逆循环 \(\displaystyle\oint\ddbar Q_{{可逆}}/T=0\) ,由于对反向循环有 \(\ddbar Q_{可逆}\to-\ddbar Q_{可逆}\) ,于是从上述定理可知 \(\ddbar Q_{可逆}/T\) 既非负又非正,即,为零。这个结果意味着任意两点$A$和$B$之间的积分 \(\ddbar Q_{可逆}/T\) 与路径无关,因为对于两条路径 (1) 和 (2) 有
		\begin{equation}\label{eq:1.25}
			\bigint_A^B\frac{\ddbar Q_{可逆}^{(1)}}{T_1}+\bigint_B^A\frac{\ddbar Q_{可逆}^{(2)}}{T_2}=0,
			\quad\implies\quad
			\bigint_B^A\frac{\ddbar Q_{可逆}^{(1)}}{T_1}=\bigint_B^A\frac{\ddbar Q_{可逆}^{(2)}}{T_2}.
		\end{equation}
		% Graphics
		\item 我们可以用\ref{eq:1.25}式构造出另一个状态函数,\termIntroduce{熵}$S$。由于该积分与路径无关而仅依赖于两端点,我们可设定
		\begin{equation}\label{eq:1.26}
			S(B)-S(A)\equiv\bigint_A^B\frac{\ddbar Q_{可逆}}{T}.
		\end{equation}
		对于可逆过程,我们现在可以用\(\ddbar Q_{可逆}=T\dd S\)来计算热量。这允许我们根据$S$为常量的条件来为一般性的(多变量的)系统构造绝热线。
		注意,\ref{eq:1.26}式所定义的熵仍可相差一个\shortRadicalTranslation{全局}{overall}常量。
		\item 对于一个可逆的(从而也是准静态的)转换,\(\ddbar Q=T\dd S\)、\(\ddbar W=\sum_iJ_i\dd x_i\)以及第一定律给出
		\begin{equation}\label{eq:1.27}
			\dd E=\ddbar Q+\ddbar W=T\dd S+\sum_iJ_i\dd x_i.
		\end{equation}
		我们可以看到$S,T$表现为一对共轭变量,$S$扮演位移而$T$则是对应的力。\longRadicalTranslation{这一观察使得我们在力学与热交换上的对应更加精妙,尽管我们应牢记温度不同于它在力学上的类比——温度是恒正的。}{This identification
		allows us to make the correspondence between mechanical and thermal exchanges more precise, although we should keep in mind that unlike its mechanical analog, temperature is always positive.}
		尽管为得到\ref{eq:1.27}式我们须诉诸于可逆转换,但值得强调的是此式是状态函数间的关系。
		\ref{eq:1.27}式可能是最基础而有用的热力学恒等式。
		\item 也可从\ref{eq:1.27}式得知描述热力学系统所需的\emph{独立变量}的数目。
		假使有$n$种方法来对一个系统做功,它们由$n$个共轭对\((J_i,x_i)\)来表示,那么就需要$n+1$个独立坐标来描述该系统。(忽略了力学坐标间的可能的约束。)
		例如选择\((E,\{x_i\})\)为坐标,由\ref{eq:1.27}式就有
		\begin{equation}
			\left.\frac{\pd S}{\pd E}\right|_{\mathbf x}=\frac{1}{T},
			\qquad
			\left.\frac{\pd S}{\pd x_i}\right|_{E,x_{j\neq i}}=-\frac{J_i}{T}.
		\end{equation}
		(其中\(\mathbf x, \mathbf J\)是参量集\(\{x_i\},\{J_i\}\)的简记。)
		\item 考虑一个由$A$至$B$的不可逆变化。用一个由$B$返回至$A$的可逆路径来构成完整循环,于是就有
		\begin{equation}
			\bigint_A^B\frac{\ddbar Q}{T}+\bigint_B^A\frac{\ddbar Q_{可逆}}{T}\leq 0,
			\quad\implies\quad
			\bigint_A^B\frac{\ddbar Q}{T}\leq S(B)-S(A).
		\end{equation}
		用微分的形式来表述的话,这意味着对于任何转换都有\(\dd S\geq \ddbar Q/T\)。特别地,现在设想绝热地隔离了多个子系统,每个子系统各自最初都处于平衡态。随着它们达到联合平衡状态,由于净值\(\ddbar Q=0\),我们必然有\(S\geq0\)。于是,由于自发的内交换至多能增大$S$而已,一个绝热系统在平衡状态下达到其熵的最大值。熵增加的方向故而给出了时间箭头与达到平衡态的路径。
		这在力学上的类比是:在表面上放置一个质点,而重力提供着一个向下的力,随着各种约束被移除,粒子将在\shortRadicalTranslation{高度更低的位置}{locations of decreasing height}安定下来。因此,有关熵增的说法并不比\bracketedText{对物体在重力作用下倾向于下落}的观察更神秘!
	\end{enumerate}



\section{接近平衡态与热力学势}

	系统朝向平衡态的时间演化由热力学第二定律来统律着。举例来说,在前面的章节中我们表明,在一个绝热隔离的系统中,熵在任何自发变化中都必须增加,并在平衡态达到最大值。那么对于非绝热隔离的非平衡态系统,它们可能还受到外部机械工作的影响呢?通常情况下,有可能定义其他的热力学势函数(thermodynamic potentials),其中在系统处于平衡态时它们取到极值。

	当没有热交换且系统通过恒定外力达到机械平衡时,焓是适当的函数。最小焓原理仅阐述了一种观察结果,即通过最小化系统的净势能加外部作用剂来获得稳定的机械平衡。

	例如,考虑一个自然长度为L0且弹簧常数为K的弹簧,受到质量为m的粒子施加的力J=mg。对于延伸到=L−L0的弹簧,其内部势能为U(x)=Kx2/2,同时粒子的势能减少了-mgx。通过将Kx2/2−mgx最小化得到机械平衡点xeq=mg/K。在其他任何位移下,弹簧最初会振荡,然后由于摩擦而停止在xeq上。对于更一般的势能函数U(x),内部产生的力Ji=−dU/dx必须在平衡点上与外力J平衡。

\section{一些有用的数学结果}
\section{稳定性条件}
\section{第三定律}
\section*{习题}