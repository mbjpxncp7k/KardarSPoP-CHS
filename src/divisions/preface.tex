\chapter{前言}

历史上,\textit{统计物理}这一学科是起源于对用物质的组成粒子来描述物质的热学性质的尝试,但它也在量子力学的发展中起到了一种基础性的作用。
更一般地,这一\shortRadicalTranslation{框架}{formalism}意在描述新行为如何从多自由度的相互作用中\shortRadicalTranslation{涌现}{emerge},进而在工程学、社会科学以及越来越多地在生物科学中产生应用。
本书介绍这门学科的中心概念与工具,并借由一系列题目、题解来将读者引向其应用。

本书所涵盖的材料直接基于我自1988年起在MIT的第一学期研究生的统计力学课程上的授课。(关联于第二学期课程的材料则由另一本配套的书\translationNote{指同作者所著的 \textit{Statistical Physics of Fields} 。}给出。)尽管这门课程的主要听众是那些第一学期的物理专业研究生,但它也往往会吸引进取的本科生以及一些其他理工科院系的学生。
这材料对统计物理书籍而言是相当\shortRadicalTranslation{中庸}{standard}的,不过听课的学生们觉得我的阐释要更有用一些,并强烈地鼓励我将这些材料出版。
本书独特的一方面在于它关于概率与相互作用粒子的两章。
% 翻译难点:interacting particles
概率是统计物理不可或缺的一部分,然而它在多数教材中并未得到足够的重视。为这个话题(以及诸如中心极限定理与信息论的相关议题)花费一整章将为读者提供许多有价值的工具。
% extensive=拓展?
在相互作用粒子的语境下,我为{\vanDerWaals}方程提供一个拓展表述,并用平均场近似给出了其推导。

做题目对学习本材料而言是必要的。我刻意选取了\shortRadicalTranslation{富有用意}{interesting}的题目(及题解),它们紧密关联于正文。
每章后都有两套题目:第一套的解答已附在本书末尾,它们意在介绍一些额外的话题、增强技术工具。
追问这些问题也应该确实对为资格考试而学习的学生有用。
而第二套题目没有提供题解,可以用于布置作业。

我最为感激的是许多我从前的学生,他们为表述材料与问题以及题解、排版文本与图形、指出笔误与谬误提供了帮助。我也向国家科学基金会的研究经费支持致谢。